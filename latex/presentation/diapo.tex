\documentclass{beamer}
 
\usepackage[utf8]{inputenc}
\usetheme{Boadilla} % Beamer theme

\title{Licence Pro ADSILLH}
 
\subtitle{GNOME-Games / GNOME-Music}
 
\author{Groupe Games:\\ Pierre Antoine Rouby,\\ Gautier Delacour,\\
  David Tabarie,\\
  \vspace{0.8cm}
  Groupe Music:\\ Kevine Carsoule,\\ Adrien Darfeuille}

\date{Année 2017/2018}

% ce code permet d'afficher le sommaire à chaque section,
% en mettant en valeur la section courante (utile ?)
\AtBeginSection[]
{
  \begin{frame}
    \frametitle{Sommaire}
    \tableofcontents[currentsection]
  \end{frame}
}

\begin{document}
\frame{\titlepage}

\section{Présentation des projets}

\begin{frame}
  \frametitle{GNOME Games}
  % add a screenshot
\end{frame}

\begin{frame}
  \frametitle{GNOME Music}
  % add a screenshot (enfin si l'envie vous prend de faire un truc, on
  % sait jamais)
\end{frame}

\section{Travaux effectués sur GNOME Games}
\subsection{Modifications du fichier 'HACKING'}
\begin{frame}
  \frametitle{Patch 1: Modifications du fichier 'HACKING'}
\end{frame}

\subsection{Ajout d'un event 'retour au menu'}
\begin{frame}
  \frametitle{Patch 2: Ajout d'un event 'retour au menu'}
\end{frame}

\section{Travaux effectués sur GNOME Music}
\subsection{Ce que vous avez fait}
\begin{frame}
  \frametitle{}
\end{frame}

\section{Conclusion}
\begin{frame}
  \frametitle{Conclusion}
\end{frame}

\end{document}
